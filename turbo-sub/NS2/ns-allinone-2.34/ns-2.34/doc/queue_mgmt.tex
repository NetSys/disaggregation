%
% personal commentary:
%        DRAFT DRAFT DRAFT
%        - KFALL
%
\chapter{Queue Management and Packet Scheduling}
\label{chap:qmgmt}

Queues represent locations where packets may be held (or dropped).
Packet scheduling refers to the decision process used to choose
which packets should be serviced or dropped.
Buffer management refers to any particular discipline used
to regulate the occupancy of a particular queue.
At present, support is included for drop-tail (FIFO) queueing,
RED buffer management, CBQ (including a priority and round-robin scheduler), 
and
variants of Fair Queueing including, Fair Queueing (FQ),
Stochastic Fair Queueing (SFQ), and Deficit Round-Robin (DRR).
In the common case where a {\em delay} element is downstream from
a queue, the queue may be {\em blocked} until it is re-enabled
by its downstream neighbor.
This is the mechanism by which transmission delay is simulated.
In addition, queues may be forcibly blocked or unblocked at arbitrary
times by their neighbors (which is used to implement multi-queue
aggregate queues with inter-queue flow control).
Packet drops are implemented in such a way that queues contain
a ``drop destination''; that is, an object that receives all packets
dropped by a queue.
This can be useful to (for example) keep statistics on dropped packets.

\section{The C++ Queue Class}
\label{sec:qclass}

The \code{Queue} class is derived from a \code{Connector} base class.
It provides a base class used by particular types of (derived) queue classes,
as well as a call-back function to implement blocking (see next section).
The following definitions are provided in \code{queue.h}:
\begin{program}
        class Queue : public Connector \{
         public:
                virtual void enque(Packet*) = 0;
                virtual Packet* deque() = 0;
                void recv(Packet*, Handler*);
                void resume();
                int blocked();
                void unblock();
                void block();
         protected:
                Queue();
                int command(int argc, const char*const* argv);
                int qlim_;         \* maximum allowed pkts in queue */
                int blocked_;
                int unblock_on_resume_; \* unblock q on idle? */
                QueueHandler qh_;
        \};
\end{program}
The \code{enque} and \code{deque} functions are pure virtual, indicating
the \code{Queue} class is to be used as a base class;
particular queues are derived
from \code{Queue} and implement these two functions as necessary.
Particular queues do not, in general, override the \code{recv} function
because it invokes the
the particular \code{enque} and \code{deque}.

The \code{Queue} class does not contain much internal state.
Often these are
\href{special monitoring objects}{Chapter}{chap:trace}.
The \code{qlim_} member is constructed to dictate a bound on the maximum
queue occupancy, but this is not enforced by the \code{Queue} class itself; it
must be used by the particular queue subclasses if they need this value.
The \code{blocked_} member is a boolean indicating whether the
queue is able to send a packet immediately to its downstream neighbor.
When a queue is blocked, it is able to enqueue packets but not send them.

\subsection{Queue blocking}
\label{sec:qblock}

A queue may be either blocked or unblocked at any given time.
Generally, a queue is blocked when a packet is in transit between it
and its downstream neighbor (most of the time if the queue is occupied).
A blocked queue will remain blocked as long as it downstream link is
busy and the queue has at least one packet to send.
A queue becomes unblocked only when its \code{resume} function is
invoked (by means of a downstream neighbor scheduling it via
a callback), usually when no packets are queued.
The callback is implemented by using the following class and
methods:
\begin{program}
        class QueueHandler : public Handler \{
        public:
                inline QueueHandler(Queue& q) : queue_(q) \{\}
                void handle(Event*); /* calls queue_.resume() */
         private:
                Queue& queue_;
        \};
        void QueueHandler::handle(Event*)
        \{
                queue_.resume();
        \}

        Queue::Queue() : drop_(0), blocked_(0), qh_(*this)
        \{
                Tcl& tcl = Tcl::instance();
                bind("limit_", &qlim_);
        \}
        void Queue::recv(Packet* p, Handler*)
        \{
                enque(p);
                if (!blocked_) \{
                        /*
                         * We're not block.  Get a packet and send it on.
                         * We perform an extra check because the queue
                         * might drop the packet even if it was
                         * previously empty!  (e.g., RED can do this.)
                         */
                        p = deque();
                        if (p != 0) \{
                                blocked_ = 1;
                                target_->recv(p, &qh_);
                        \}
                \}
        \}
        void Queue::resume()
        \{
                Packet* p = deque();
                if (p != 0)
                        target_->recv(p, &qh_);
                else \{
                        if (unblock_on_resume_)
                                blocked_ = 0;
                        else
                                blocked_ = 1;
                \}
        \}
\end{program}
The handler management here is somewhat subtle.
When a new \code{Queue} object is created,
it includes a \code{QueueHandler} object (\code{qh_})
which is initialized to contain a reference to the new \code{Queue} object
(\code{Queue& QueueHandler::queue_}).
This is performed by the \code{Queue} constructor using the expression
\code{qh_(*this)}.
When a Queue receives a packet it calls the subclass
(i.e. queueing discipline-specific) version of
the \code{enque} function with the packet.
If the queue is not blocked, it is allowed to send a packet and
calls the specific \code{deque} function which determines which
packet to send, blocks the queue (because a packet is now in transit), and
sends the packet to the queue's downstream neighbor.
Note that any future packets received from upstream neighbors
will arrive to a blocked queue.
When a downstream neighbor wishes to cause the queue to become unblocked
it schedules the QueueHandler's \code{handle} function by
passing \code{&qh_} to the simulator scheduler.
The \code{handle} function invokes \code{resume}, which
will send the next-scheduled packet downstream (and leave the queue blocked),
or unblock the queue when no packet is ready to be sent.
This process is made more clear by also referring to the
\href{\fcn[]{LinkDelay::recv} method}{Section}{sec:delayclass}.

\subsection{PacketQueue Class}
\label{sec:packetqclass}

The \code{Queue} class may implement buffer management and scheduling but
do not implement the low-level operations on a particular queue.
The \code{PacketQueue} class is used for this purpose, and is defined as follows
(see \code{queue.h}):
\begin{program}
        class PacketQueue \{
        public:
                PacketQueue();
                int length(); /* queue length in packets */
                void enque(Packet* p);
                Packet* deque();
                Packet* lookup(int n);
                /* remove a specific packet, which must be in the queue */
                void remove(Packet*);
        protected:
                Packet* head_;
                Packet** tail_;
                int len_;               // packet count
        \};
\end{program}
This class maintains a linked-list of packets, and is commonly
used by particular scheduling and buffer management disciplines
to hold an ordered set of packets.
Particular scheduling or buffer management schemes may make
use of several \code{PacketQueue} objects.
The \code{PacketQueue} class maintains current counts of the number of
packets held in the queue which is returned by the \fcn[]{length} method.
The \code{enque} function places the specified packet at the end of
the queue and updates the \code{len_} member variable.
The \code{deque} function returns the packet at the head of the
queue and removes it from the queue (and updates the counters), or
returns NULL if the queue is empty.
The \code{lookup} function returns the $n$th packet from the head
of the queue, or NULL otherwise.
The \code{remove} function deletes the packet stored in the given address
from the queue (and updates the counters).
It causes an abnormal program termination if the packet does not exist.

\section{Example: Drop Tail}
\label{sec:droptail}

The following example illustrates the implementation of the
\code{Queue/DropTail} object,
which implements FIFO scheduling and
drop-on-overflow buffer management typical of most present-day
Internet routers.
The following definitions declare the class and its OTcl linkage:
\begin{program}
        /*
         * {\cf A bounded, drop-tail queue}
         */
        class DropTail : public Queue \{
         protected:
                void enque(Packet*);
                Packet* deque();
                PacketQueue q_;
        \};
\end{program}
The base class \code{Queue},
from which \code{DropTail} is derived, provides most
of the needed functionality.
The drop-tail queue maintains exactly one FIFO queue, implemented
by including an object of the \code{PacketQueue} class.
Drop-tail implements its own versions of \code{enque} and \code{deque}
as follows:
\begin{program}
        /*
         * {\cf drop-tail}
         */
        void DropTail::enque(Packet* p)
        \{
                q_.enque(p);
                if (q_.length() >= qlim_) \{
                        q_.remove(p);
                        drop(p);
                \}
        \}

        Packet* DropTail::deque()
        \{
                return (q_.deque());
        \}
\end{program}
Here, the \code{enque} function first stores the packet in the
internal packet queue (which has no size restrictions), and then
checks the size of the packet queue versus \code{qlim_}.
Drop-on-overflow is implemented by dropping the packet most recently
added to the packet queue if the limit is reached or exceeded.
\emph{Note:} in the implementation of \code{enque} above, 
 setting \code{qlim_} to \code{n} actually means a queue size of \code{n-1}.
Simple FIFO scheduling is implemented in the \code{deque} function
by always returning the first packet in the packet queue.


\section{Different types of Queue objects}
\label{sec:queueobjects}
A queue object is a general class of object capable of holding and
possibly marking or discarding packets as they travel through the
simulated topology. Configuration Parameters used for queue objects are:
\begin{description}
\item[limit\_] The queue size in packets. 

\item[blocked\_] Set to false by default, this is true if the queue is
blocked (unable to send a packet to its downstream neighbor). 

\item[unblock\_on\_resume\_] Set to true by default, indicates a queue
should unblock itself at the time the last packet packet sent has been
transmitted (but not necessarily received). 
\end{description}

Other queue objects derived from the base class Queue are drop-tail, FQ,
SFQ, DRR, RED and CBQ queue objects. Each are described as follows:
\begin{itemize}

\item Drop-tail objects:
Drop-tail objects are a subclass of Queue objects that implement simple
FIFO queue. There are no methods, configuration parameter, or state
variables that are specific to drop-tail objects. 

\item FQ objects:
FQ objects are a subclass of Queue objects that implement Fair queuing.
There are no methods that are specific to FQ objects. 
Configuration Parameters are:
\begin{description}
\item[secsPerByte\_] 
\end{description}
There are no state variables associated with this object. 

\item SFQ objects:
SFQ objects are a subclass of Queue objects that implement Stochastic Fair
queuing. There are no methods that are specific to SFQ objects. 
Configuration Parameters are:
\begin{description}
\item[maxqueue\_]

\item[buckets\_]
\end{description}
There are no state variables associated with this object. 

\item DRR objects:
DRR objects are a subclass of Queue objects that implement deficit round
robin scheduling. These objects implement deficit round robin scheduling
amongst different flows ( A particular flow is one which has packets with
the same node and port id OR packets which have the same node id alone).
Also unlike other multi-queue objects, this queue object implements a
single shared buffer space for its different flows. Configuration
Parameters are:
\begin{description}
\item[buckets\_] Indicates the total number of buckets to be used for
hashing each of the flows. 

\item[blimit\_] Indicates the shared buffer size in bytes. 

\item[quantum\_] Indicates (in bytes) how much each flow can send during
its turn. 

\item[mask\_] mask\_, when set to 1, means that a particular flow consists
of packets having the same node id (and possibly different port ids),
otherwise a flow consists of packets having the same node and port ids. 
\end{description}

\item RED objects:
RED objects are a subclass of Queue objects that implement random
early-detection gateways. The object can be configured to either drop or
``mark'' packets. There are no methods that are specific to RED objects. 
Configuration Parameters are:
\begin{description}
\item[bytes\_] Set to "true" to enable ``byte-mode'' RED, where the size
of arriving packets affect the likelihood of marking (dropping) packets. 

\item[queue-in-bytes\_]
Set to "true" to measure the average queue size in bytes rather than
packets. Enabling this option also causes thresh\_ and maxthresh\_ to be
automatically scaled by mean\_pktsize\_ (see below). 

\item[thresh\_]
The minimum threshold for the average queue size in packets. 

\item[maxthresh\_]
The maximum threshold for the average queue size in packets. 

\item[mean\_pktsize\_]
A rough estimate of the average packet size in bytes. Used in updating the
calculated average queue size after an idle period. 

\item[q\_weight\_]
The queue weight, used in the exponential-weighted moving average for
calculating the average queue size. 

\item[wait\_]
Set to true to maintain an interval between dropped packets. 

\item[linterm\_]
As the average queue size varies between "thresh\_" and "maxthresh\_", the
packet dropping probability varies between 0 and "1/linterm". 

\item[setbit\_]
Set to "true" to mark packets by setting the congestion indication bit in
packet headers rather than drop packets. 

\item[drop-tail\_]
Set to true to use drop-tail rather than randomdrop when the queue
overflows or the average queue size exceeds "maxthresh\_". For a further
explanation of these variables, see [2]. 
\end{description}
None of the state variables of the RED implementation are accessible. 


\item CBQ objects:
CBQ objects are a subclass of Queue objects that implement class-based
queueing. 

\code{$cbq insert <class>}\\
Insert traffic class class into the link-sharing structure associated with
link object cbq. 

\code{$cbq bind <cbqclass> <id1> [$id2]}\\
Cause packets containing flow id id1 (or those in the range id1 to
id2 inclusive) to be associated with the traffic class cbqclass. 

\code{$cbq algorithm <alg>}\\
Select the CBQ internal algorithm. <alg> may be set to one of:
"ancestor-only", "top-level", or "formal". 


\item CBQ/WRR objects:
CBQ/WRR objects are a subclass of CBQ objects that implement weighted
round-robin scheduling among classes of the same priority level. In
contrast, CBQ objects implement packet-by-packet round-robin scheduling
among classes of the same priority level. Configuration Parameters are:
\begin{description}
\item[maxpkt\_] The maximum size of a packet in bytes. This is used only
by CBQ/WRR objects in computing maximum bandwidth allocations for the
weighted round-robin scheduler. 
\end{description}
\end{itemize}


\textsc{CBQclass Objects}\\
CBQClass objects implement the traffic classes associated with CBQ
objects. 

\code{$cbqclass setparams <parent> <okborrow> <allot> <maxidle> <prio> <level>}\\
Sets several of the configuration parameters for the CBQ traffic class
(see below). 

\code{$cbqclass parent <cbqcl|none>}\\
specify the parent of this class in the link-sharing tree. The parent may
be specified as ``none'' to indicate this class is a root. 

\code{$cbqclass newallot <a>}\\
Change the link allocation of this class to the specified amount (in range
0.0 to 1.0). Note that only the specified class is affected. 

\code{$cbqclass install-queue <q>}\\
Install a Queue object into the compound CBQ or CBQ/WRR link structure.
When a CBQ object is initially created, it includes no internal queue
(only a packet classifier and scheduler).

Configuration Parameters are:
\begin{description}
\item[okborrow\_] is a boolean indicating the class is permitted to borrow
bandwidth from its parent. 

\item[allot\_] is the maximum fraction of link bandwidth allocated to the
class expressed as a real number between 0.0 and 1.0. 

\item[maxidle\_] is the maximum amount of time a class may be required to
have its packets queued before they are permitted to be forwarded 

\item[priority\_]
is the class' priority level with respect to other classes. This value may
range from 0 to 10, and more than one class may exist at the same
priority. Priority 0 is the highest priority. 

\item[level\_]
is the level of this class in the link-sharing tree. Leaf nodes in the
tree are considered to be at level 1; their parents are at level 2, etc. 

\item[extradelay\_]
increase the delay experienced by a delayed class by the specified time 
\end{description}


\textsc{Queue-monitor objects}\\
QueueMonitor Objects are used to monitor a set of packet and byte arrival,
departure and drop counters. It also includes support for aggregate
statistics such as average queue size, etc.

\code{$queuemonitor}\\
reset all the cumulative counters described below (arrivals, departures,
and drops) to zero. Also, reset the integrators and delay sampler, if
defined. 

\code{$queuemonitor set-delay-samples <delaySamp_>}\\
Set up the Samples object delaySamp\_ to record statistics about queue
delays. delaySamp\_ is a handle to a Samples object i.e the Samples object
should have already been created. 

\code{$queuemonitor get-bytes-integrator}\\
Returns an Integrator object that can be used to find the integral of the
queue size in bytes. 

\code{$queuemonitor get-pkts-integrator}\\
Returns an Integrator object that can be used to find the integral of the
queue size in packets.

\code{$queuemonitor get-delay-samples}\\
Returns a Samples object delaySamp\_ to record statistics about queue
delays.
\\
There are no configuration parameters specific to this object. 
\\
State Variables are:
\begin{description}
\item[size\_] Instantaneous queue size in bytes. 

\item[pkts\_] Instantaneous queue size in packets. 

\item[parrivals\_] Running total of packets that have arrived. 

\item[barrivals\_] Running total of bytes contained in packets that have
arrived. 

\item[pdepartures\_] Running total of packets that have departed (not
dropped). 

\item[bdepartures\_] Running total of bytes contained in packets that have
departed (not dropped). 

\item[pdrops\_] Total number of packets dropped. 

\item[bdrops\_] Total number of bytes dropped. 

\item[bytesInt\_] Integrator object that computes the integral of the
queue size in bytes. The sum\_ variable of this object has the running sum
(integral) of the queue size in bytes. 

\item[pktsInt\_] Integrator object that computes the integral of the queue
size in packets. The sum\_ variable of this object has the running sum
(integral) of the queue size in packets. 
\end{description}


\textsc{QUEUEMONITOR/ED Objects}\\
This derived object is capable of differentiating regular packet drops
from early drops. Some queues distinguish regular drops (e.g. drops due to
buffer exhaustion) from other drops (e.g. random drops in RED queues).
Under some circumstances, it is useful to distinguish these two types of
drops. 
\\
State Variables are:
\begin{description}
\item[epdrops\_] The number of packets that have been dropped ``early''. 

\item[ebdrops\_] The number of bytes comprising packets that have been
dropped ``early''. 
\end{description}

Note: because this class is a subclass of QueueMonitor, objects of this
type also have fields such as pdrops\_ and bdrops\_. These fields describe
the total number of dropped packets and bytes, including both early and
non-early drops. 


\textsc{QueueMonitor/ED/Flowmon Objects}\\
These objects may be used in the place of a conventional QueueMonitor
object when wishing to collect per-flow counts and statistics in addition
to the aggregate counts and statistics provided by the basic QueueMonitor. 

\code{$fmon classifier <cl>}\\
This inserts (read) the specified classifier into (from) the flow monitor
object. This is used to map incoming packets to which flows they are
associated with. 

\code{$fmon dump}\\
Dump the current per-flow counters and statistics to the I/O channel
specified in a previous attach operation. 

\code{$fmon flows}\\
Return a character string containing the names of all flow objects known
by this flow monitor. Each of these objects are of type
QueueMonitor/ED/Flow. 

\code{$fmon attach <chan>}\\
Attach a tcl I/O channel to the flow monitor. Flow statistics are written
to the channel when the dump operation is executed. 

Configuration Parameters are:
\begin{description}
\item[enable\_in\_] Set to true by default, indicates that per-flow
arrival state should be kept by the flow monitor. If set to false, only
the aggregate arrival information is kept. 

\item[enable\_out\_]
Set to true by default, indicates that per-flow departure state should be
kept by the flow monitor. If set to false, only the aggregate departure
information is kept. 

\item[enable\_drop\_]
Set to true by default, indicates that per-flow drop state should be kept
by the flow monitor. If set to false, only the aggregate drop information
is kept. 

\item[enable\_edrop\_]
Set to true by default, indicates that per-flow early drop state should be
kept by the flow monitor. If set to false, only the aggregate early drop
information is kept. 
\end{description}


\textsc{QueueMonitor/ED/Flow Objects}\\
These objects contain per-flow counts and statistics managed by a
QueueMonitor/ED/Flowmon object. They are generally created in an OTcl
callback procedure when a flow monitor is given a packet it cannot map on
to a known flow. Note that the flow monitor's classifier is responsible
for mapping packets to flows in some arbitrary way. Thus, depending on the
type of classifier used, not all of the state variables may be relevant
(e.g. one may classify packets based only on flow id, in which case the
source and destination addresses may not be significant). 
State Variables are:
\begin{description}
\item[src\_] The source address of packets belonging to this flow. 

\item[dst\_] The destination address of packets belonging to this flow. 

\item[flowid\_] The flow id of packets belonging to this flow. 
\end{description}

\section{Commands at a glance}
\label{sec:queuecommand}

Following is a list of queue commands used in simulation scripts:
\begin{flushleft}
\code{$ns_ queue-limit <n1> <n2> <limit>}\\
This sets a limit on the maximum buffer size of the queue in the link between
nodes <n1> and <n2>.


\code{$ns_ trace-queue <n1> <n2> <optional:file>}\\
This sets up trace objects to log events in the queue. If tracefile is not
passed, it uses \code{traceAllFile_} to write the events.


\code{$ns_ namtrace-queue <n1> <n2> <optional:file>}\\
Similar to trace-queue above, this sets up nam-tracing in the queue.


\code{$ns_ monitor-queue <n1> <n2> <optional:qtrace> <optional:sampleinterval>}\\
This command inserts objects that allows us to monitor the queue size. This
returns a handle to the object that may be queried to determine the average
queue size. The default value for sampleinterval is 0.1.


\end{flushleft}

% JoBS contributed by Nicolas Christin <nicolas@cs.virginia.edu>
\section{Queue/JoBS}
\label{sec:jobsx}

\emph{JoBS is developed and contributed by Nicolas Christin <nicolas@cs.virginia.edu>}

This chapter describes the implementation of the Joint Buffer Management 
and Scheduling (JoBS) algorithm in {\em ns}. This chapter is in three parts. 
The first part summarizes the objectives of the JoBS algorithm. The second 
part explains how to configure a JoBS queue in {\em ns}. The third part 
focuses on the tracing mechanisms implemented for JoBS. 

The procedures and functions described in this chapter can be found in 
\ns/jobs.\{cc, h\}, 
\ns/marker.\{cc, h\}, 
\ns/demarker.\{cc, h\}. 
Example scripts can be found 
in \ns/tcl/ex/jobs-\{lossdel, cn2002\}.tcl. 

\noindent Additional information 
can be found at http://qosbox.cs.virginia.edu.

\subsection{The JoBS algorithm}
This section gives an overview of the objectives the JoBS algorithm
aims at achieving, and of the mechanisms employed to reach these
objectives. The original JoBS algorithm, as described in
\cite{LiCh01}, was using the solution to a non-linear optimization
problem. This {\em ns-2} implementation uses the feedback-control
based heuristic described in \cite{ChLiAb02a}.

\noindent{\em Important Note:} 
This {\em ns-2} implementation results from the 
merge between old code for {\em ns-2.1b5}, and code derived from the 
BSD kernel-level 
implementation of the JoBS algorithm. {\bf It is still considered experimental.}
Due to the absence of binding facilities for arrays 
between Tcl and C++ in {\em tclcl} at the moment, {\em the number of 
traffic classes is 
statically set to 4 and cannot be changed without modifying the C++ code.}

\subsubsection{Objective}
The objective of the JoBS algorithm is to provide absolute and
relative (proportional) loss and delay differentiation independently
at each node for {\em classes} of traffic. JoBS therefore provides
service guarantees on a {\em per-hop} basis.  The set of performance
requirements are specified to the algorithm as a set of per-class
Qualtiy of Service (QoS) constraints.  As an example, for three
classes, the QoS constraints could be of the form:
\begin{itemize}
\item{$\mbox{Class-1 Delay} \approx 2 \cdot  \mbox{Class-2 Delay}$}, 
\item{$\mbox{Class-2 Loss Rate}  \approx 10^{-1} \cdot\mbox{Class-3 Loss Rate}$, or} 
\item{$\mbox{Class-3 Delay} \leq 5~ms$.} 
\end{itemize}
Here, the first two constraints are relative constraints and the last 
one is an absolute constraint. The set of constraints can be 
any mix of relative and absolute constraints. 
More specifically, JoBS supports the five following types of constraints:
\begin{itemize}
\item{{\bf Relative delay constraints (RDC)} specify a proportional 
delay differentiation between classes. As an example, for two classes $1$ and $2$, the RDC enforces a relationship
\[
\frac{\mbox{Delay of Class 2}}{\mbox{Delay of Class 1}}\approx \mbox{constant}\ .\nonumber
\]
}
\item{{\bf Absolute delay constraints (ADC)}: An ADC on class~$i$ requires that the delays of class~$i$ satisfy a worst-case bound $d_i$.}
\item{{\bf Relative loss constraints (RLC)} specify a proportional 
loss differentiation between classes.}
\item{{\bf Absolute loss constraints (ALC)}: An ALC on class~$i$ requires that the loss rate of class~$i$ be bounded by an upper bound $L_i$.}
\item{{\bf Absolute rate constraints (ARC)}: An ARC on class~$i$ means that the throughput of class~$i$ is bounded by a lower bound $\mu_i$.}
\end{itemize}

JoBS does not rely on admission control or traffic policing, nor does it 
make any assumption on traffic arrivals. 
Therefore, a system of constraints may become infeasible, and 
some constraints may need to be relaxed. 
QoS constraints are prioritized in the following order.
\[ 
\mbox{ALC} > \mbox{ADC, ARC} > \mbox{Relative Constraints} \ .
\]
That is, if JoBS is unable to satisfy both absolute and relative constraints, 
it will give preference to the absolute constraints.

\subsubsection{Mechanisms}
JoBS performs scheduling and buffer management in a 
single pass. JoBS dynamically allocates service rates to 
classes in order to satisfy the delay constraints. The service rates needed 
for enforcing absolute delay constraints 
are allocated upon each packet arrival, 
while service rates derived from 
relative delay constraints are computed only 
every $N$ packet arrivals. If no feasible 
service rate allocation exists\footnote{For instance, 
if the sum of the service 
rates needed is greater than the output link capacity.}, 
or if the packet buffer overflows, 
packets are dropped according to the loss constraints. 

The service rates are translated into packet scheduling decisions by 
an algorithm resembling Deficit Round Robin. That is, the scheduler 
tries to achieve the desired service rates by keeping track of the 
difference between the actual transmission rate for each class and the 
desired service rate for each class. Scheduling in JoBS is work-conserving.

\subsection{Configuration}
Running a JoBS simulation requires to create and configure the JoBS 
``link(s)'', to create and configure the Markers and Demarkers in charge 
of marking/demarking the traffic, to attach an application-level data 
source (traffic generator), and to start the traffic generator.

\subsubsection{Initial Setup}
\begin{program}
set ns [new Simulator] \; preamble initialization;

{\bfseries{}Queue/JoBS set drop_front_ false} \; use drop-tail;
{\bfseries{}Queue/JoBS set trace_hop_ true} \; enable statistic traces;
{\bfseries{}Queue/JoBS set adc_resolution_type_ 0} \; see ``commands at a glance'';
{\bfseries{}Queue/JoBS set shared_buffer_ 1} \; all classes share a common buffer;
{\bfseries{}Queue/JoBS set mean_pkt_size_ 4000}\; we expect to receive 500-Byte pkts;
{\bfseries{}Queue/Demarker set demarker_arrvs1_ 0}\; reset arrivals everywhere;
{\bfseries{}Queue/Demarker set demarker_arrvs2_ 0}
{\bfseries{}Queue/Demarker set demarker_arrvs3_ 0}
{\bfseries{}Queue/Demarker set demarker_arrvs4_ 0}
{\bfseries{}Queue/Marker set marker_arrvs1_ 0}
{\bfseries{}Queue/Marker set marker_arrvs2_ 0}
{\bfseries{}Queue/Marker set marker_arrvs3_ 0}
{\bfseries{}Queue/Marker set marker_arrvs4_ 0}

set router(1) [$ns node] \; set first router;
set router(2) [$ns node] \; set second router;
set source [$ns node] \; set source;
set sink [$ns node] \; set traffic sink;
set bw 10000000 \; 10 Mbps;
set delay 0.001 \; 1 ms;
set buff 500 \; 500 packets;
\end{program}
\subsubsection{Creating the JoBS links}
\begin{program}
{\bfseries{}$ns duplex-link $router(1) $router(2) $bw $delay JoBS} \; Creates the JoBS link;
$ns_ queue-limit $router(1) $router(2) $buff
set l [$ns_ get-link $router(1) $router(2)]
set q [$l queue]
{\bfseries{}$q init-rdcs -1 2 2 2} \; Classes 2, 3 and 4 are bound by proportional delay differentiation with a factor of 2;
{\bfseries{}$q init-rlcs -1 2 2 2} \; Classes 2, 3 and 4 are bound by proportional loss differentiation with a factor of 2;
{\bfseries{}$q init-alcs 0.01 -1 -1 -1} \; Class 1 is provided with a loss rate bound of 1\%;
{\bfseries{}$q init-adcs 0.005 -1 -1 -1} \; Class 1 is provided with a delay bound of 5 ms;
{\bfseries{}$q init-arcs -1 -1 -1 500000} \; Class 4 is provided with a minimumthroughput of 500 Kbps;
{\bfseries{}$q link [$l link]} \; The link is attached to the queue (required);
{\bfseries{}$q trace-file jobstrace} \; Trace per-hop, per-class metrics to the file jobstrace;
{\bfseries{}$q sampling-period 1} \; Reevaluate rate allocation upon each arrival;
{\bfseries{}$q id 1}\; Assigns an ID of 1 to the JoBS queue;
{\bfseries{}$q initialize}\; Proceed with the initialization;
\end{program}

\subsubsection{Marking the traffic}
Marking the traffic is handled by Marker objects. Markers are FIFO queues that 
set the class index of each packet. To ensure accuracy of the simulations, 
it is best to configure these queues to have a very large buffer, so that no 
packets are dropped in the Marker. Demarkers are used to gather end-to-end 
delay statistics.
\begin{program}
{\bfseries{}$ns_ simplex-link $source $router(1) $bw $delay Marker} \; set-up marker;
$ns_ queue-limit $source $router(1) [expr $buff*10] \; Select huge buffers for markers;
$ns_ queue-limit $router(1) $source [expr $buff*10] \; to avoid traffic drops; 
set q [$ns_ get-queue $source $router(1)] \;in the marker;
{\bfseries{}$q marker_type 2} \; Statistical marker;
{\bfseries{}$q marker_frc 0.1 0.2 0.3 0.4} \; 10\% Class 1, 20\% Class 2, 30\% Class 3, 40\% Class 4.;
{\bfseries{}$ns_ simplex-link $router(2) $sink $bw $delay Demarker} \; set-up demarker;
$ns_ queue-limit $router(2) $sink [expr $buff*10] 
{\bfseries{}$q trace-file e2e} \; trace end-to-end delays to file e2e;
\end{program}
The remaining steps (attaching agents and traffic generators or applications 
to the nodes) are explained in 
Chapters~\ref{sec:agents} and \ref{chap:applications}, and are handled as 
usual. We refer to these chapters and the example scripts provided with your 
{\em ns} distribution. 

\subsection{Tracing}
Tracing in JoBS is handled internally, by the scheduler. 
Each JoBS queue can generate a 
trace file containing the following information.  
Each line of the tracefile consists of 17 columns. The first column is the 
simulation time, columns 2 to 5 represent the loss rates over the current busy 
period for classes 1 to 4, columns 6 to 9 represent the delays for each class 
(average over a 0.5 seconds sliding window), columns 10 to 13 represent the 
average service rates allocated to each class over the last 0.5 seconds, and 
columns 14 to 17 represent the instantaneous queue length in packets. 
Additionally, Demarkers can be used to trace end-to-end delays.

\subsection{Variables}
This section summarizes the variables that are used by JoBS, Marker and Demarker objects.
\subsubsection{JoBS objects}
\begin{description}
\item[trace\_hop\_] Can be true or false. If set to true, per-hop, per-class metrics will be traced. (Trace files have then to be specified, using \code{<JoBS object> trace-file <filename>}.) Defaults to false.
\item[drop\_front\_] Can be true or false. If set to true, traffic will be dropped from the front of the queue. Defaults to false (drop-tail).
\item[adc\_resolution\_type\_] Can be 0 or 1. 
If set to 0, traffic will be dropped from classes that have an ADC if the ADC 
cannot be met by adjusting the service rates. If set to 1, traffic will be 
dropped from all classes. A resolution mode set to 
1 is therefore fairer, in the sense that the pain is shared by all classes, 
but can lead to more deadline violations. Defaults to 0.
\item[shared\_buffer\_] Can be 0 or 1. If set to 0, all classes use a separate 
per-class 
buffer (which is required if only rate guarantees are to provided). All 
per-class buffers have the same size. 
If set to 
1, all classes share the same buffer (which is required if loss differentiation
is to be provided). Defaults to 1.
\item[mean\_pkt\_size\_] 
Used to set the expected mean packet size of packets arriving at a JoBS link. 
Setting this variable is required to ensure proper delay differentiation.
\end{description}
\subsubsection{Marker objects}
\begin{description}
\item[marker\_arrvs1\_] 
Number of Class-1 packets to have entered a Marker link.
\item[marker\_arrvs2\_] 
Number of Class-2 packets to have entered a Marker link.
\item[marker\_arrvs3\_] 
Number of Class-3 packets to have entered a Marker link.
\item[marker\_arrvs4\_] 
Number of Class-4 packets to have entered a Marker link.
\end{description}
\subsubsection{Demarker objects}
\begin{description}
\item[demarker\_arrvs1\_] 
Number of Class-1 packets to have entered a Demarker link.
\item[demarker\_arrvs2\_] 
Number of Class-2 packets to have entered a Demarker link.
\item[demarker\_arrvs3\_] 
Number of Class-3 packets to have entered a Demarker link.
\item[demarker\_arrvs4\_] 
Number of Class-4 packets to have entered a Demarker link.
\end{description}

\subsection{Commands at a glance}
The following is a list of commands used to configure the JoBS, Marker and 
Demarker objects.

\subsubsection{JoBS objects}
\begin{flushleft}
\code{set q [new Queue/JoBS]}\\
This creates an instance of the JoBS queue. 

\code{$q init-rdcs <k1> <k2> <k3> <k4>}\\
This assigns the RDCs for the four JoBS classes. For instance, using a value 
of 4 for k2 means that Class-3 delays will be roughly equal to four times 
Class-2 delays. A value of -1 indicates that 
the class is not concerned by RDCs. 

\noindent{\em Important Note:} Since RDCs bound two classes, one would 
expect only three parameters to be passed (k1, k2, and k3, since k4 
theoretically binds Classes 4 and 5, and Class~5 does not exist). 
However, in this prototype implementation, it is imperative to specify a value 
different from 0 and -1 to k4 if Class~4 is to be concerned by RDCs. 

\noindent{\em Examples:} \code{$q init-rdcs -1 2 1 -1} specifies that classes~2 and 3 
are bound by a delay differentiation factor of 2, \code{$q init-rdcs 4 4 4 4} 
specifies that all classes are bound by a delay differentiation factor of 4 and
is equivalent to \code{$q init-rdcs 4 4 4 1}, since the last coefficient is 
only used to specify that Class~4 is to be bound by proportional 
differentiation.

\code{$q init-rlcs <k'1> <k'2> <k'3> <k'4>}\\
This assigns the RLCs for the four JoBS classes. 
For instance, using a value 
of 3 for k1 means that Class-2 loss rates will be roughly equal to four times 
Class-2 loss rates.
A value of -1 indicates that 
the class is not concerned by RLCs. As with RDCs, each RLC binds two classes, 
thus, one would 
expect only three parameters to be passed (k'1, k'2, and k'3, since k'4 
theoretically bounds Classes 4 and 5, and Class~5 does not exist). 
As explained above, it is imperative to specify a value 
different from 0 and -1 to k'4 if Class~4 is to be concerned by RLCs.

\code{$q init-alcs <L1> <L2> <L3> <L4>}\\
This assigns the absolute loss guarantees (ALCs) to all four classes. L1 to L4 
are given in fraction of 1. For instance, setting L1 to 0.05 means that Class-1
loss rate will be guarantees to be less than 5\%. A value of -1 indicates that 
the corresponding class is not subject to an ALC.

\code{$q init-adcs <D1> <D2> <D3> <D4>}\\
This assigns the absolute loss guarantees (ADCs) to all four classes. D1 to D4 
are given in milliseconds. A value of -1 indicates that 
the corresponding class is not subject to an ADC.

\code{$q trace-file <filename>}\\
This specifies the trace file for all per-hop metrics. JoBS uses an internal 
module to trace
loss and delays, service rates, and per-class queue lengths in packets. 
If filename is set to {\bf null}, no trace will be provided.

\code{$q link [<link-object> link]}\\
This command is required to bind a link to a JoBS queue. Note that JoBS needs
to know the capacity of the link. Thus, 
this command {\bf has to} be issued before 
the simulation is started.

\code{$q sampling-period <sampling-interval>}\\
This command specifies the sampling interval (in packets) at which the service 
rate adjustments for proportional differentiation will be performed. The 
default is a sampling interval of 1 packet, meaning that the rate allocation 
is reevaluated upon each packet arrival. Larger sampling intervals speed up 
the simulations, but typically result in poorer proportional differentiation.

\code{$q id <num_id>}\\
This command affects a numerical ID to the JoBS queue. 

\code{$q initialize}\\
This command is required, and should be run after all configuration operations 
have been performed. 
This command will perform the final checks and configuration of the JoBS queue.

\code{$q copyright-info}\\ 
Displays authors and copyright information.

A simple example script (with nam output), fully annotated and commented 
can be found in \ns/tcl/ex/jobs-lossdel.tcl. 
A more realistic example of a simulation with JoBS queues can be found in 
\ns/tcl/ex/jobs-cn2002.tcl. This script is very similar to what was used in 
a simulation presented in \cite{LiCh02}. 
Associated tracefiles and {\em gnuplot} scripts for visualization (in case you 
favor {\em gnuplot} over {\em xgraph}
can be found in \ns/tcl/ex/jobs-lossdel, and \ns/tcl/ex/jobs-cn2002.


\end{flushleft}

\subsubsection{Marker objects}
\code{$q marker_type <1|2>}\\
Selects the type of marker. 1 is DETERMINISTIC, 2 is STATISTICAL. 

\code{$q marker_class <1|2|3|4>}\\
For a deterministic marker, selects which class packets should be marked with.

\code{$q marker_frc <f1> <f2> <f3> <f4>}\\
For a statistical marker, gives the fraction of packets that should be marked 
from each class. For instance, using 0.1 for f1 means that 10 percent of the 
traffic coming to the Marker link will be marked as Class 1.

\subsubsection{Demarker objects}
\code{$q trace-file <filename>}\\
This command specifies the trace file used for the demarker object. 
filename.1 will contain the end-to-end 
delays of each Class-1 packet to have reached the 
Demarker link, filename.2 will contain the end-to-end 
delays of each Class-2 packet to have reached the 
Demarker link, and so forth. (There will of course be 4 trace files, one 
for each class.)

\endinput

### Local Variables:
### mode: latex
### comment-column: 60
### backup-by-copying-when-linked: t
### file-precious-flag: nil
### End:


\endinput
